\documentclass{article}

\usepackage{graphics}
\usepackage[vcentering]{geometry}

\geometry{papersize={6.9in,2.8in}, top=0.05in, bottom=0in, left=0in, right=0in}

\input{tikzlib.tex}
\input{diagram_geom.tex}

\definecolor{c1}{RGB}{217, 95, 2}
\definecolor{c2}{RGB}{117, 112, 179}
\definecolor{c3}{RGB}{231, 41, 138}
\definecolor{c4}{RGB}{102, 166, 30}

\usetikzlibrary{arrows.meta}

\begin{document}

\begin{tikzpicture}
\node(local)[bigcompartment, text width=5cm, bottom color=c4]{
    \textbf{Locally corrected GI}
    \footnotesize{
        \begin{itemize}
        \item based on degree distribution and close contact rate [3]
        \item does not explicitly take contact structures into account
        \end{itemize}
    }
};
\node(int)[bigcompartment, below=2.4cm of local, text width=5cm, bottom color=c1]{
    \textbf{Intrinsic GI}
    \footnotesize{
        \begin{itemize}
        \item patient based (e.g. titer)
        \end{itemize}
    }
};
\draw[->, very thick] (int) -- (local);
\node(eff)[bigcompartment, right=3.4cm of local, text width=6cm, bottom color=c2]{
    \textbf{Effective GI}
    \begin{itemize}
    \item accounts for spatial (network) structure
    \item equivalent to intrinsic GI under homogeneous mixing
    \end{itemize}
};
\draw[->, very thick, dashed] ($(int.north)+(2.7,0)$) -- node [midway, above, sloped] {Difficult in practice} ($(eff.south)+(-3.1,0)$); 
\node(obs)[bigcompartment, below=2cm of eff, text width=6cm, bottom color=c3]{
    \textbf{Observed GI in early epidemic}
    \begin{itemize}
    \item contact-tracing based
    \item proportional to $g(\tau) \exp(-r\tau)$
        \end{itemize}
};
\draw[->, very thick] (obs.north) -- node [text width=5cm, midway, right] {Exponential correction\\ \footnotesize{weight observed periods by $\exp(r\tau)$}} (eff);
\node(R)[right=of eff]{$\mathcal{R}$};
\draw[->, very thick] (eff) -- (R);
\draw[->, very thick] (obs) -- node [text width=2cm, above]{Homogeneous\\ assumption}(int);
% \node(patient)[above left=0cm and 0.2cm of int, rotate=90]{Patient based};
\end{tikzpicture}

\end{document}
